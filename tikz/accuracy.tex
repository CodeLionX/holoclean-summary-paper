% !TeX root = accuracy.tex
% !TeX encoding = UTF-8
% !TeX spellcheck = en_US

\documentclass[crop,tikz]{standalone}
\usetikzlibrary{calc}
\usepackage{pgfplots}

\makeatletter
\@ifundefined{holoclean}{%
  \newcommand{\holoclean}{HoloClean}%
}{}
\makeatother

\begin{document}

% accuracy diagram
\newcommand{\plotcolor}{black}
\begin{tikzpicture}
  \begin{axis}[
      ybar,
      symbolic x coords={Hospital,Flights,Food,Physicians},
      xtick style={draw=none},
      xtick pos=left,
      axis x line*=bottom,
      ytick pos=left,
      bar width=7,
      legend style={at={(0.5,-0.2)}, anchor=north, legend columns=-1},
      ymin=0, ymax=1,
      ylabel={Accuracy ($F_1$)},
      %scale only axis,
      %width=0.8\linewidth
      enlarge x limits=0.2,
      cycle list = {\plotcolor, \plotcolor!70, \plotcolor!40, \plotcolor!10},
    ]
    \addplot+[fill,draw=black] coordinates {
      (Hospital,0.832)
      (Flights,0.763)
      (Food,0.783)
      (Physicians,0.897)
    };
    \addplot+[fill,draw=black] coordinates {
      (Hospital,0.435)
      (Flights,0.0)
      (Food,0.235)
      (Physicians,0.512)
    };
    \addplot+[fill,draw=black] coordinates {
      (Hospital,0.379)
      (Flights,0.0)
      (Food,0.473)
      (Physicians,0.0)
    };
    \addplot+[fill,draw=black] coordinates {
      (Hospital,0.593)
      (Flights,0.104)
      (Food,0.0)
      (Physicians,0.0)
    };
    \legend{\holoclean{},Holistic,KATARA,SCARE};
    
    \def\ymin{\pgfkeysvalueof{/pgfplots/ymin}}
    \def\ymax{\pgfkeysvalueof{/pgfplots/ymax}}
    \def\drawSep#1#2{%
    \draw[black,-]
      ($(axis cs:{#1}, \ymin)!.5!(axis cs:{#2}, \ymin)$) --
      ($(axis cs:{#1}, \ymax)!.5!(axis cs:{#2}, \ymax)$);
    }
    \drawSep{Hospital}{Flights}
    \drawSep{Flights}{Food}
    \drawSep{Food}{Physicians}
  \end{axis}
\end{tikzpicture}

\end{document}