% !TeX root = holoclean.tex
% !TeX encoding = UTF-8
% !TeX spellcheck = en_US

\section{Data Repairing}\label{sec:introduction}
  Data is a central aspect in most enterprises.
  It is the basis for operational and strategic business decision making.
  However, real world data is flawed and contains errors which can impair reporting, customer service and production facilitation.
  Studies show that poor data quality leads to costs of billions of dollars~\cite{Redman:quality_disaster, cost_of_low_qual}.
  Maintaining the quality and consistency of business data has therefore become a critical task.

  One way to improve data quality is data cleaning.
  It consists of two steps: Error detection and data repairing.
  Error detection describes to process of identifying incorrect values in a dataset.
  Subsequently data repairing transforms the erroneous dataset into a new one removing detected errors, so the new dataset adheres to data quality requirements.

  \bigskip
  \textbf{Related Work:}
  Many researchers have made efforts to automate the task of error detection. \textbf{TBD:}
  \begin{itemize}
    \item Most of the approaches try to detect the violation of \glspl{ic}.
    \item duplicate detection
    \item outlier detection
  \end{itemize}

  Data repairing techniques can be classified according to whether and how humans are involved in the repairing process.
  On the one end there are fully automatic approaches like SCARE that leverage machine learning and intelligent partition algorithms to repair a database without the need of user input~\cite{scare}.
  On the other end there are data wrangling tools that make extensive use of human knowledge to clean a dirty dataset.
  For example, Data Wrangler~\cite{data_wrangler}, the successor of Potter's Wheel~\cite{potters_wheel}, provides a visual interface for cleaning a dataset.
  Based on data statistics it creates visual suggestions for data transformations.
  The user selects and executes those transformations to create a cleaned dataset instance.
  The tool can export the transformation sequence, to repeat the process on another dataset.
  This makes it an appropriate tool for the manual definition of ETL processes.
  The final version of the tool, Trifacta Wrangler~\cite{trifacta_wrangler}, is developed commercially.

  All those tools~\cite{scare,potters_wheel,data_wrangler,trifacta_wrangler} use quantitative statistics of the dataset itself to repair errors in it.
  Besides that, state-of-the-art methods also use \glspl{ic}~\cite{ajax,gdr,editing_rules,data_tamer} and external knowledge~\cite{katara}, like dictionaries, knowledge bases or domain expert annotations, as input signals.
  Figure~\ref{fig:tools} shows those tools aligned on their level of user interaction needed to perform data repairing and the usage of the three different input signals.

  \begin{figure}[hbt]
    \centering
    \begin{tikzpicture}[
        xscale=4.2, yscale=4.2,>=triangle 60,
        tool/.style={
          circle,
          draw=black,
          fill=black!30,
          inner sep=0pt,
          minimum size=25pt
        }
      ]
      \small
      \newcommand{\fillColor}{black!25}
      \newcommand{\drawColor}{black}
      \newcommand{\circleSize}{0.2}
      
      
      \begin{scope}
      \draw[black,<->] (-1.25,-0.25) -- (1.25,-0.25) ;
      \draw[black,<->] (-1.25,-0.25) -- (-1.25,1.25) ;
      
      \draw[tool] (0,1) ellipse (1.15 and 0.15) node {HoloClean\cite{holoclean}};
      
      \node[tool, align=center] at (-1,-0.11) {Data\\Tamer\cite{data_tamer}};
      \node[tool, align=center] at (-1,0.14) {Editing\\Rules\cite{editing_rules}};
      \node[tool, align=center] at (-1,0.53) {Guided\\Data\\Repair\cite{gdr}};
      \node[tool, align=center] at (-1,0.9) {AJAX\\\cite{ajax}};
      
      \node[tool, align=center] at (0,0) {KATARA\\\cite{katara}};
      
      \node[tool, align=center] at (0.73,-0.05) {Potter's\\Wheel\cite{potters_wheel}};
      \node[tool, align=center] at (0.95,0.12) {Data\\Wrangler\\\cite{data_wrangler}};
      \node[tool, align=center] at (1.15,-0.06) {Trifacta\\Wrangler\\\cite{trifacta_wrangler}};
      
      \node[tool, align=center] at (1,1.05) {SCARE\\\cite{scare}};
      
      \node[align=center] at (-1.5,1) {Automatic};
      \node[align=center] at (-1.5,0) {Human\\Guided};
      
      \node[align=center] at (-1,-0.4) {Integrity\\Constraints};
      \node[align=center] at (0,-0.4) {External\\Knowledge};
      \node[align=center] at (1,-0.4) {Statistics\\about Data};
      
      \end{scope}
    \end{tikzpicture}

    \caption{Data Repairing Tools in Context}
    \label{fig:tools}
  \end{figure}

  \bigskip
  Most of the tools limit themselves to only one signal to perform data repairing, ignoring the other information.
  Each type of signal is associated with a different action on the dataset and has its own downsides.
  Data repairs that use \glspl{ic} could introduce incorrect repairs, because they are relying on the \textit{minimality} principle and assume that most of the data values are clean.
  This is not necessarily the case and minimal repairs not always correspond to correct repairs~\cite{holoclean}.
  Repair algorithms relying on external information are dependent on the coverage of the external data source and can therefore perform poorly.
  Quantitative statistics heavily depend on the available information in the dataset itself.
  For small datasets and unfortunate situations this can drastically decrease repairing quality.
  
  \begin{itemize}
    \item combination of methods --> Holistic approach would be beneficial to use all available information.
    \item Combining heterogeneous signals is challenging as they could suggest conflicting repairs.
    \item \citeauthor{holoclean} propose a holistic data repairing approach that includes all aforementioned signals~\cite{holoclean}: HoloClean
    \item HoloClean, overview
    \item HoloClean, what is different? (abgrenzung zu anderen verfahren)
    \item Paper Preview/Overview
  \end{itemize}


\section{Background}\label{sec:background}

\section{The HoloClean Framework}\label{sec:framework}

\section{Performance of HoloClean}\label{sec:performance}

\section{Conclusion}\label{sec:conclusion}