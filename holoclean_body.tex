% !TeX root = holoclean.tex
% !TeX encoding = UTF-8
% !TeX spellcheck = en_US

\section{Data Repairing}\label{sec:introduction}
  Data is a central aspect in most enterprises.
  It is the basis for operational and strategic business decision making.
  However, real world data is flawed and contains errors which can impair reporting, customer service and production facilitation.
  Studies show that poor data quality leads to costs of billions of dollars~\cite{Redman:quality_disaster, cost_of_low_qual}.
  Maintaining the quality and consistency of business data has therefore become a critical task.

  One way to improve data quality is data cleaning.
  It consists of two steps: Error detection and data repairing.
  Error detection describes to process of identifying incorrect values in a dataset.
  Subsequently data repairing transforms the erroneous dataset into a new one removing detected errors, so the new dataset adheres to data quality requirements.
  
  \textbf{Related Work:}
  
  now techniques and classification (automation, human-guided, input)
  
  \begin{figure}[hbt]
    \centering
    \begin{tikzpicture}[
        xscale=4.5, yscale=4.5,>=triangle 60,
        tool/.style={
          circle,
          draw=black,
          fill=black!30,
          inner sep=0pt,
          minimum size=25pt
        }
      ]
      \newcommand{\fillColor}{black!25}
      \newcommand{\drawColor}{black}
      \newcommand{\circleSize}{0.2}
      
      
      \begin{scope}
        \draw[black,<->] (-1.25,-0.25) -- (1.25,-0.25) ;
        \draw[black,<->] (-1.25,-0.25) -- (-1.25,1.25) ;
        
        \draw[tool] (0,1) ellipse (1.15 and 0.15) node {HoloClean\cite{rekatsinas2017holoclean}};
        
        \node[tool, align=center] at (-1,-0.11) {Data\\Tamer\cite{data_tamer}};
        \node[tool, align=center] at (-1,0.11) {Editing\\Rules\cite{editing_rules}};
        \node[tool, align=center] at (-1,0.5) {Guided\\Data\\Repair\cite{gdr}};
        \node[tool, align=center] at (-1,0.9) {AJAX\\\cite{ajax}};
        
        \node[tool, align=center] at (0,0) {KATARA\\\cite{katara}};
        
        \node[tool, align=center] at (0.76,-0.08) {Potter's\\Wheel\cite{potters_wheel}};
        \node[tool, align=center] at (0.95,0.06) {Data\\Wrangler\\\cite{data_tamer}};
        \node[tool, align=center] at (1.15,-0.11) {Trifacta\\Wrangler\\\cite{trifacta_wrangler}};
        
        \node[tool, align=center] at (1,1.05) {SCARE\\\cite{scare}};
        
        \node[align=center] at (-1.5,1) {Automatic};
        \node[align=center] at (-1.5,0) {Human\\Guided};
        
        \node[align=center] at (-1,-0.4) {Integrity\\Constraints};
        \node[align=center] at (0,-0.4) {External\\Knowledge};
        \node[align=center] at (1,-0.4) {Statistics\\about Data};
      
      \end{scope}
    \end{tikzpicture}
    
    \caption{Data Repairing Tools in Context}
    \label{fig:tools}
  \end{figure}

  \begin{itemize}
    \item HoloClean, overview
    \item HoloClean, what is different? (abgrenzung zu anderen verfahren)
    \item Paper Preview/Overview
  \end{itemize}
  

\section{Background}\label{sec:background}

\section{The HoloClean Framework}\label{sec:framework}

\section{Performance of HoloClean}\label{sec:performance}

\section{Conclusion}\label{sec:conclusion}